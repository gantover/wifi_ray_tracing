\begin{lstlisting}[language=python]
from utils import *

P_MAX_40 = 1e-7  # = -40dbm
P_MAX_50 = 1e-8  # = -50 dbm
P_MIN = 1e-12  # = -90 dbm
B_MAX = 40.0  # binary debit in [GB/s]
B_MIN = 50 * 1e-3

P_MAX_CL = 1 / ((12.5 * 1e-3)**2)
# P_MAX_CL is the maximum power normalized by PRX0 (at the limit of
# the distant fields hypothesis)

PTX = 0.1  # emitter power [W]

Z0 = 120 * np.pi  # empty space impedance
EPS0 = 8.85418782e-12
MU0 = 4 * np.pi * 1e-7
C = 1.0 / np.sqrt(EPS0 * MU0)

FREQ = 6e10  # working frequency of 60Ghz

OMEGA = 2.0 * np.pi * FREQ
BETA0 = OMEGA * np.sqrt(MU0 * EPS0)
RAR = 73  # Emission Resistor (we neglect losses)
LAMBDA = C / FREQ
GP = (Z0 * PTX) / (np.pi * RAR)  # Grx * Prx
PRX0 = (LAMBDA**2 * 60 * GP) / (8 * RAR * np.pi**2)

# floor dimensions
x_size = 15  # [m]
y_size = 8  # [m]


@ti.data_oriented
class Dimensions:
    # containing relevant dimension data here allows for an easy update from
    # anywhere in the code
    def __init__(self, x_size, y_size, cell_size):
        self.x_size = x_size
        self.y_size = y_size
        self.cell_size = cell_size
        self.unit_step_density = 1.0/cell_size
        self.x = int(self.unit_step_density * self.x_size)
        self.y = int(self.unit_step_density * self.y_size)

    def update(self, cell_size):
        self.cell_size = cell_size
        self.unit_step_density = 1.0/cell_size
        self.x = int(self.unit_step_density * self.x_size)
        self.y = int(self.unit_step_density * self.y_size)


dim = Dimensions(x_size, y_size, 0.2)

n = 1  # number of emitters

# to use with Rays
tx = vec2([9.4, 1.0])
# rx = vec2([8.0, 6.0])
rx = vec2([2.0, 5.0])

"""Sample exercise data"""
# PTX = 1e-3
# FREQ = 868.3e6
#
# OMEGA = 2.0 * np.pi * FREQ
# BETA0 = OMEGA * np.sqrt(MU0 * EPS0)
# LAMBDA = C / FREQ
# GP = (Z0 * PTX) / (np.pi * RAR)
# PRX0 = (LAMBDA**2 * 60 * GP) / (8 * RAR * np.pi**2)
#
# x_size = 80
# y_size = 90
# cell_size = 0.5
# unit_step_density = (1/cell_size)
# x_dimension = int(unit_step_density * x_size)
# y_dimension = int(unit_step_density * y_size)
# rx = vec2([47.0, 65.0])
# tx = vec2([32.0, 10.0])

\end{lstlisting}